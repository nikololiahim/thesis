\begin{abstract}
    В данной диссертации описывается пробная реализация статического анализатора для
    объектно-ориентированных программ, которые обнаруживают два дефекта семейства "хрупких базовых классов" - 
    непредвиденную взаимную рекурсию и неоправданное допущение в модификаторе (подклассе).
    Реализация анализаторов опирается на промежуточное представление, называемое
    EO (сокращение от Elegant Objects), которое основано на $\varphi$-исчислении - 
    формализации общей семантики объектно-ориентированных языков программирования, вдохновленной шаблоном проектирования "декоратор". Правильность
    анализаторов была проверена с использованием подхода тестирования на основе свойств, 
    а также рукописных модульных тестов для выявления важных краевых случаев.
    
\end{abstract}