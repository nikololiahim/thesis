\chapter{Implementation}
\label{chap:impl}
This chapter analyzes specifics of Odin (short for Object Dependency INspector) - a static analyzer of EO source code. Section 4.1 covers the tools and technologies used in the project. Section 4.2 gives a brief overview of the project file structure.
Section 4.3 goes over the implementation of EO parser used in the project. Section 4.4 discusses the ways we can represent the elements of object-oriented programs in EO. Sections 4.5 and 4.6 describe the implementations of the analysis algorithms applied to structured EO code. Finally, section 4.7 summarizes this chapter.

\section{Development Environment}
Odin is a project written entirely in \textbf{Scala} \footnote{\url{https://www.scala-lang.org/}} - a modern programming language with support for high-level concepts such as structural pattern matching \cite{pattern_matching} and algebraic data types \cite{adts}. Scala is compiled into Java Virtual Machine (JVM) byte code. This allows Odin to be used as a library in any other project compatible with JVM, be it Scala or Java. In addition, programs compiled to JVM byte code can be run without changes on any device that can run Java Virtual Machine.


The project uses a build tool called \textbf{sbt} \footnote{\url{https://www.scala-sbt.org/}}, which allows compiling multiple Scala modules at once. A distinctive feature of \textbf{sbt} is the ability to cross-compile Scala code so that it is compatible with many versions of Scala and Java. It also supports a variety of plugins that improve the development process. The plugins used by Odin are \textbf{scalafmt} \footnote{\url{https://scalameta.org/scalafmt/}}, an automatic source code formatter, and \textbf{scalafix} \footnote{\url{https://scalacenter.github.io/scalafix/}} , a linter and code analyzer with support for project-wide refactorings.

Odin is published as a \textbf{JAR} \footnote{\url{https://docs.oracle.com/javase/7/docs/technotes/guides/jar/jar.html}} and can be downloaded from the \textbf{Maven Central} repository \footnote{\url{https://search.maven.org/search?q=g:org.polystat.odin}}.

The source code of the project is available on \textbf{Github} \footnote{\url{https://github.com/polystat/odin}}. It also provides the instructions on how to launch and contribute to the project.



\section{Module Structure}
Odin is a project consisting of multiple modules. The main modules are:
\begin{itemize}
    \item Core, which contains the definition for EO AST (Abstract Syntax Tree).
          This AST is used as an input to all analysis algorithms.
    \item Analyses, which contains the implementations of various analyzers.
    \item Backends, which contains algorithms that transform EO AST into something else. The only backend so far is a plain text backend: it transforms EO AST into its syntactically correct equivalent in EO source code. This backend can also be interpreted as a pretty-printer of EO code and is widely used as such in other modules.
    \item Parser, which contains a parser (also known as a syntactic analyzer) of EO source code. It is used to convert different EO representations (e.g. plain text or XML encoding) into the EO AST defined in Core module.
\end{itemize}

\section{Data Structures}
\subsection{EO Syntax Tree}
This is a data structure that is used to model the syntactic structure of EO which is used as a starting point for the extraction of more high-level concepts, such as class-objects, method-objects and method calls. It describes EO following the syntax specification from the paper by Bugayenko \cite{eolang}, with slight deviations to account for the specifics of the underlying refined $\varphi$-calculus described in \cite{kudasov}. 

In order to create the parser from the EO code to EO syntax tree we used \textbf{cats-parse} \footnote{\url{https://github.com/typelevel/cats-parse}}, a monadic parser combinator \cite{hill_combinators_1996}  library for Scala.

EO syntax tree is an immutable polymorphic data structure defined \textit{à la carte} \cite{alacarte} \pic{fig:ast}. 
Since the tree is immutable, it can only be altered by constructing the new version, where the old parts of the tree are replaced with new ones. The transformations that construct these new versions of the data structure are known as \textit{optics} \cite{optics}. We used the \textit{monocle} \footnote{\url{https://www.optics.dev/Monocle/}} library for Scala to simplify the generation of the optics that modify the EO syntax tree.  

\begin{figure}
    \lstinputlisting[language=Scala]{code/ast.scala}
    \caption{EO syntax tree definitions (abridged).}
    \label{fig:ast}
\end{figure}
\subsection{Object Tree}
Object Tree is a data strcuture that captures the relationships between objects in an EO program. It is a refinenement of the EO syntax tree, which contains the elements of an EO program relevant to subsequent analysis steps: class-objects, method-objects, extension clauses and method calls. 

EO object tree is also a recursively-defined polymorphic data structure \pic{fig:objtree}. The type parameter $A$ represents the information that is stored for each object in the tree. This information is stored in \textit{info} field of the tree. The field called \textit{nestedObjs} is stores the information about all the nested class-objects. Nested objects are the class-objects that are defined as the attributes of other class objects, just like nested classes in Java. The information about one of the nested objects can be accessed by the key which is of type \textit{Name}. This name identifies the object uniquely because the object can not contain two attributes with the same name.  

\begin{figure}
    \begin{lstlisting}
        final case class ObjectTree[A](
          info: A,
          nestedObjs: Map[Name, ObjectTree[A]]
        ) 
    \end{lstlisting}
    \caption{Object tree}
    \label{fig:objtree}
\end{figure}

\subsection{ObjectInfo}

The first and the most generally-applicable type we use in place of type parameter $A$  is \textit{ObjectInfo} \pic{fig:objinfo}. This type also has two type parameters. The first one, $P$, is responsible for holding the information about the decorated object (or simply parent). The first traversal can only gather the name of the decorated object. 

The second type parameter, $M$, is the type used to store the information about each of the methods. The information captured during the first traversal of EO syntax tree \pic{fig:methodinfo} can be summarised as follows:
\begin{itemize}
    \item \textit{selfArgName} - the name of the free attribute of the method object that is used to capture the calling object. 
    \item \textit{body} - the EO syntax tree node that hold the body of the method. 
    \item \textit{depth} - how deeply the method object is nested in an EO program. For toplevel objects this attribute is $0$. For method objects (that are always defined in the class-objects), this attribute is equal to the depth of the class-object plus one.  
    \item \textit{calls} - a sequence of method calls in the method definition. The \textit{Call} type stores all the necessary information to identify and traverse all the call within the method body:
    \begin{itemize}
        \item depth - depth of the object where the call is located. This value is equal to the depth of the method-object + the relative depth of the method-local object containing the call.
        \item \textit{methodName} - a simple name of the method-object where the call is located.
        \item \textit{callSite} - an optic \cite{optics} which extracts the location of the method-local object where the call is located. 
        \item \textit{callLocation} - an optic \cite{optics} that extracts the location of the EO syntax tree node that defines the method call. 
        \item \textit{args} - the EO syntax tree nodes, which correspond to the arguments of the call, including the \textbf{self} argument.  
    \end{itemize}  
\end{itemize}

\begin{figure}
    \begin{lstlisting}[language=Scala]
      final case class ObjectInfo[P, M](
        name: Name,
        fqn: FQName,
        depth: Int,
        parentInfo: Option[P],
        methods: Map[Name, M],
      )
    \end{lstlisting}
    \caption{ObjectInfo}
    \label{fig:objinfo}
\end{figure}

\begin{figure}
    \begin{lstlisting}[language=Scala]
      final case class MethodInfo(
          selfArgName: String,
          calls: Vector[Call],
          body: EOObj[EOExprOnly],
          depth: BigInt,
      )
          
      final case class Call(
        depth: BigInt,
        methodName: String,
        callSite: PathToCallSite,
        callLocation: PathToCall,
        args: NonEmptyVector[EOBnd[EOExprOnly]]
      )
    \end{lstlisting}
    \caption{MethodInfo and Call}
    \label{fig:methodinfo}
\end{figure}

\section{Partial object tree}
The \textit{ObjectInfo}, where $P$ is the name of the parent object and $M$ is \textit{MethodInfo} can be called the \textit{partial object tree}:
\begin{lstlisting}
    type PartialObjectTree = ObjectTree[
        ObjectInfo[ParentName, MethodInfo]
    ]

\end{lstlisting} 



\section{Detecting Unanticipated Mutual Recursion}
\label{impl:mutualrec}

\subsection{Proposed solution}
The solution to the problem lies in detecting the cycles in the call-graphs of all the objects. For each class-object in the program, do the following:
\begin{enumerate}
    \item Detect the decorated class-object, all methods, and for each method in the class detect all the methods it calls. If the method that is called exists in the class-object, mark it as \textit{resolved}. Otherwise, mark it as \textit{partially-resolved}. The set of mappings between the methods of the class and the methods that each of the methods calls is considered a \textit{partial call-graph} of the object.
    \item After that the tree is traversed again to convert all the partially-resolved calls to fully resolved calls. To do that we need to calculate the \textit{complete call-graph} of the object, which contains the methods from the object itself, as well as the methods from the decorated object. This is done by \textit{extending} the partial call-graph of the decorated object with the partial call-graph of the decorating objects. Hereinafter we use the terms \textbf{child} and \textbf{parent} to refer to the decorating object and the decorated object respectively. The extension procedure is defined as follows:
          \begin{enumerate}
              \item if the method is present in the parent call-graph, but is absent in the child call-graph, it is left as is.
              \item if the method is present in the child call-graph but does not exist in the parent call-graph, it is added to the parent call-graph.
              \item if the method is present both in the child call-graph and the parent call-graph, all the occurrences of the method in the parent call-graph are replaced by the child's version of the method.
          \end{enumerate}
    \item After the object's call-graph is resolved, perform the depth-first search \cite{dfs} to find the cycles in the complete call-graph. After all the cycles are found, exclude the cycles that contain only the methods from the same object.
\end{enumerate}

\section{Detecting Unjustified Assumption in Subclass}
\label{impl:unjustified}

\subsection{Proposed Solution}
We propose the following approach for detecting the methods where inlining of the calls may lead to breaking changes in subclasses:
\begin{enumerate}
    \item An \textit{initial} representation of the program is produced. This representation is a tree-like data structure which preserves the nesting relations between objects. So, the objects which contain other objects are the roots of their respective subtrees, whereas the container objects are the subtrees or leaves.
    \item We produce a \textit{revision} of the initial program representation where all the calls to the methods are inlined.
    \item In both versions, for each of the class-objects, for each method in the class-object, a set of \textit{properties} is inferred.  These properties can be thought of as an implicit contract \cite{meyer} of each method. In addition to the implicit properties, the explicit properties which come in the form of \textit{assert} statements in the source code are also taken into account. In order to infer the properties of the method, partial interpretation of its body is performed. The interpretation is limited to basic numeric operations, numeric and boolean values and method calls. The inference rules are described in greater detail in fig. \ref{fig:props}.
    \item After all the properties are inferred, the following predicate should hold true for both the initial and the revised versions:
          \begin{align*}
              P_{init} \implies P_{rev}
          \end{align*}
          If it doesn't hold for some class-object, it means that the revision of one of its superclasses introduces a breaking change, which weakens the precondition of some its methods.
\end{enumerate}

\begin{figure}
    \includegraphics[width=\textwidth]{figs/properties}
    \caption{Rules for property inference in detection of unjustified assumption in subclass.}
    \label{fig:props}
\end{figure}

