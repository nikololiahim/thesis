\chapter{Заключение}
\label{chap:conclusion}

В настоящее время парадигма объектно-ориентированного программирования является одним из наиболее доминирующих инструментов в арсенале компаний-разработчиков программного обеспечения. Попытки угнаться за постоянно растущим спросом на инновации привели к неизбежному росту сложности программного обеспечения. Объектно-ориентированные кодовые базы, пожалуй, больше всего страдают от этой сложности, порождая такие явления, как "унаследованный код"\cite{legacy}. Существует множество подходов к решению этой проблемы. Один из наиболее часто используемых называется статическим анализом - рассуждениями о коде без его выполнения. 

В данной диссертации описывается инновационный подход к статическому анализу объектно-ориентированных программ с использованием $\varphi$-calculus - формализации семантики объектно-ориентированных программ, основанной на идее паттерна декоратора \cite{GOFPatterns}. Мы изучили существующие работы по $\varphi$-calculus и его реализации, EO \cite{eolang}, описали один из вариантов $\varphi$-calculus и применили его для построения статического анализатора, обнаруживающего проблемы семейства "хрупких базовых классов" \cite{fragilebaseclass}. Реализация была задокументирована, а исходный код опубликован в открытом репозитории Github.

Анализатор был всесторонне протестирован на собственноручно написанных примерах. В настоящее время в конвейере непрерывной интеграции выполняется 89 тестов, некоторые из которых приближаются к 700 строкам кода EO. 

\section{Вклад этой работы}
\begin{itemize}
    \item Парсер для подмножества грамматик EO, описанных в \cite{eolang}.
    \item Набор полезных структур данных для анализа программ, переведенных в промежуточное представление EO. 
    \item Два алгоритма, использующие вышеуказанные структуры данных для обнаружения двух дефектов семейства "хрупких базовых классов" - непредвиденной взаимной рекурсии и неоправданного допущения в подклассе.
\end{itemize}

\section{Будущая работа}
Была проделана значительная работа, однако остаются проблемы, которые необходимо решить, прежде чем анализаторы, использующие подход, описанный в данной диссертации, смогут находить дефекты в объектно-ориентированных кодовых базах промышленного масштаба. 

\begin{itemize}
    \item Анализаторы, описанные в данной диссертации, должны быть соединены с транслятором с целевого языка (например, Java) в промежуточное представление EO. Нам известно о нескольких существующих работах в этой области, однако было проделано мало работы, чтобы заставить их работать вместе с анализаторами.
    \item Выполнение сложных анализов, таких как выявление необоснованного предположения в подклассе, значительно затруднено из-за отсутствия в EO средства проверки типов. Наличие информации о типах объектов во время анализа сделало бы процесс вывода ограничений, подобный описанному в \ref{impl:unjustified_algo}, более надежным и точным. 
    \item Анализаторы успешно работают на небольших тестовых примерах, однако их тестирование на больших массивах кода EO, например, сгенерированных из существующего исходного кода Java, может выявить критические недостатки текущего дизайна, а также узкие места в производительности, которые трудно обнаружить на небольших тестовых примерах. 
    \item Текущая кодировка, используемая для перевода элементов объектно-ориентированных программ в EO \ref{lit:encoding}, хотя и является достаточно общей для представления простых объектно-ориентированных программ, не отражает всей полноты возможностей, присутствующих в объектно-ориентированных языках программирования. Необходимо провести дополнительные исследования, чтобы разработать более полную кодировку. 
    \item Текущий дизайн анализаторов не учитывает сообщения об ошибках, поэтому их текущие возможности ограничены полностью вычисленными именами объектов, в которых произошли ошибки. Необходимо пересмотреть дизайн, чтобы сообщения об ошибках указывали на точные места ошибок в промежуточном представлении EO и, соответственно, в исходном коде целевого языка.
\end{itemize}

